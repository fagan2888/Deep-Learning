\documentclass{article}
\usepackage[utf8]{inputenc}

\title{Deep Learning DS-GA 1008  Assignment2}

\author{Xiao Jing xj655}
\date{February 2019}

\usepackage{natbib}
\usepackage{graphicx}

\begin{document}

\maketitle

\section{Convolution}
\item (a)
What is the dimensionality of the output if we forward propagate the image over the given convolution kernel with no padding and stride of 1?

Answer: The dimension is 3x3 \\

\item (b)
Give a general formula of the output width O in terms of the input width I, kernel width K, stride S, and padding P (both in the beginning and in the end).\\

Answer:
$$
O = \frac{I-K+2P}{S} +1
$$
In the example of (a), I=5, K=3, P=0, S=1.
We can get the same answer that the dimension O=3.\\

\item (c)
Compute the output C of forward propagating the image over the given convolution kernel. Assume that the bias term of the convolution is zero.

Answer:\\

\begin{tabular}{|c |c| c|}
     \hline
      109	& 92 & 72\\
      \hline
       108	& 85 & 74\\
       \hline
        110	& 74 & 79\\
        \hline
\end{tabular}

\item (d) 
Suppose the gradient backpropagated from the layers above this layer is a 3 × 3 matrix of all 1s. Write the value of the gradient (w.r.t. the input image) backpropagated out of this layer.\\

Answer:

Assume $a_i_j\in A$,and $i,j\in$  $\{1,2,\cdot\cdot\cdot,5\}$
$$
\frac{\partial E}{\partial A} = \frac{\partial E}{\partial C}\frac{\partial C}{\partial A}
$$

Since for each scalar
$$
\frac{\partial E}{\partial c_i_j} = 1
$$
So $\frac{\partial E}{\partial C}$ is a 3x3 matrix\\
\begin{tabular}{|c |c| c|}
     \hline
       1 & 1 & 1\\
      \hline
       1 & 1 & 1\\
       \hline
       1 & 1& 1 \\
        \hline
\end{tabular}\\

For each $$\frac{\partial E}{\partial a\textsubscript{ij}} = 1*\frac{\partial C}{\partial a\textsubscript{ij}}$$

For each $\frac{\partial C}{\partial a\textsubscript{ij}}$,is to figure out how many weights in the kernel apply on the $a\textsubscript{ij}$

Since 
$$
\frac{\partial C}{\partial A} =\left[
\begin{array}{c c c c}
    \frac{\partial C}{\partial a\textsubscript{11}}&\frac{\partial C}{\partial a\textsubscript{12}}&\cdots&\frac{\partial C}{\partial a\textsubscript{15}} \\
    \frac{\partial C}{\partial a\textsubscript{21}}&\frac{\partial C}{\partial a\textsubscript{22}}&\cdots&\frac{\partial C}{\partial a\textsubscript{25}} \\
    \vdots &\vdots &\ddots&\vdots \\
    \frac{\partial C}{\partial a\textsubscript{51}}&\frac{\partial C}{\partial a\textsubscript{52}}&\cdots&\frac{\partial C}{\partial a\textsubscript{55}}
\end{array}
\right]\\
=\left[
\begin{array}{c c c c c}
    4&7&10&6&3&\\
    9&17& 25& 16& 8\\
    11&23&34&22& 11\\
    4&12& 18& 12& 6\\
    2&6& 9&6& 3
\end{array}
\right]
$$
So, the backpropagation of this layer is 
$$
\left[
\begin{array}{c c c c c}
    4&7&10&6&3&\\
    9&17& 25& 16& 8\\
    11&23&34&22& 11\\
    4&12& 18& 12& 6\\
    2&6& 9&6& 3
\end{array}
\right]
$$


\section{Pooling}
\item  (a)
\\
1. torch.nn.MaxPool2d--- will pick the maximum value  over each  kernel size area from the input. In regular max pooling, you downsize an input set by taking the maximum value of smaller N x N subsections of the set (often 2x2), and try to reduce the set by a factor of N, where N is an integer.\\

2. torch.nn.AvgPool2d---will take the average of value over each kernel size  area from the input \\

3. torch.nn.LPPool2d---Learn-norm Pooling will apply a 2D power-average pooling over an input signal composed of several inputplanes.\\

4. torch.nn.AdaptiveAvgPool2d---Applies a adaptive average pooling over an input signal composed of several input planes.\\

5.torch.nn.FractionalMaxPool2d---Slightly differenct from maxpooling. The overall reduction ratio N does not have to be an integer.The sizes of the pooling regions are generated randomly but are fairly uniform.


\item  (b)
\\
1. Max Pooling 2d, when $S^k_{i,j}$ is the indices of $X^k$.
\begin{equation}
    Y^k_{i,j}  = \max(X^k_in)=\max (X^k[S^k_{i,j}])
\end{equation}

2. Average Pooling 2d
\begin{equation}
    Y^k_{i,j}  = average(X^k_in) = average(X^k[S^k_{i,j}])
\end{equation}

3. LP-Pooling 2d
\begin{equation}
        Y^k_{i,j} = \sqrt[p]{\sum_{x \in X^k[S^k_{i,j}]k} x^{p}}
\end{equation}
- x is the elements in sub-region $X^k$\\
- At $p = \infty, f(X)$ gets Max Pooling\\
- At p = 1, one gets Sum Pooling (which is proportional to Average Pooling)
\\\\
\item  (c)Answer:
\\\\
\begin{tabular}{|c |c|}
     \hline
      109	& 92 \\
      \hline
       110	& 85\\
        \hline
\end{tabular}
\\\\
\item  (d)

In LP pooling\\
When power p = 1, one gets Sum Pooling (which is proportional to Average Pooling)
\begin{equation}
        Y^k_{i,j} = \sqrt[p]{\sum_{x \in X^k[S^k_{i,j}]} x^{p}} = \sqrt[1]{\sum_{x \in X^k[S^k_{i,j}]} x} = \sum_{x \in X^k[S^k_{i,j}]} x = 
        S^k_{i,j}\text{count} \cdot average(X^k[S^k_{i,j}])\\ 
        =\text{count}\{S^k_{i,j}\}\cdot Y^k_{avg,ij}
\end{equation}
\\
When $p = \infty$,
\begin{equation}
        Y^k_{i,j} = \sqrt[p]{\sum_{x \in X^k[S^k_{i,j}]} x^{p}} \approx \sqrt[p]{\max_{x \in X^k[S^k_{i,j}]} (x^{p})} = \max(X^k[S^k_{i,j}])
\end{equation}

\end{document}
